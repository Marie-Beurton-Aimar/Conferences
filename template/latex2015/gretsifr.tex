%% ==============================================================
%% WARNING! ENGLISH SPEAKING AUTHORS SHOULD READ gretsien.tex
%%          FILE INSTEAD
%% ==============================================================
%% EXEMPLE DE CONTRIBUTION AU GRETSI
%% POUR LES UTILISATEURS FRANCOPHONES DE LaTeX2e
  \documentclass{gretsi}
%% Selectionnez ensuite les paquets que vous utilisez,
%% par suppression ou adjonction d'un caractere %
%% en debut de ligne (mise en commentaire).
%% --------------------------------------------------------------
%% UTILISATION DE CARACTERES ACCENTUES AU CLAVIER ?
%% (le codage du clavier depend du systeme d'exploitation)
% \usepackage[applemac]{inputenc} % MacOS
% \usepackage[ansinew]{inputenc}  % Windows ANSI
% \usepackage[cp437]{inputenc}    % DOS, page de code 437
% \usepackage[cp850]{inputenc}    % DOS, page de code 850
% \usepackage[cp852]{inputenc}    % DOS, page de code 852
% \usepackage[cp865]{inputenc}    % DOS, page de code 865
% \usepackage[latin1]{inputenc}   % UNIX, codage ISO 8859-1
% \usepackage[decmulti]{inputenc} % VMS
% \usepackage[next]{inputenc}
% \usepackage[latin2]{inputenc}
% \usepackage[latin3]{inputenc}
%% --------------------------------------------------------------
%% REGLES DE TYPOGRAPHIE FRANCAISES ?
 \usepackage[english,french]{babel}   % "babel.sty" + "french.sty"
% \usepackage[english,francais]{babel} % "babel.sty"
% \usepackage{french}                  % "french.sty"
  \usepackage{times}			% ajout times le 30 mai 2003
 
%% --------------------------------------------------------------
%% CODAGE DE POLICES ?
%% Si votre moteur Latex est francise, il est conseille
%% d'utiliser le codage de police T1 pour faciliter la c�sure,
%% si vous disposez de ces polices (DC/EC)
% \usepackage[T1]{fontenc}
%% ==============================================================

\titre{Instructions aux auteurs du GRETSI\\ Format \LaTeXe}

\auteur{\coord{Michel}{Dupont}{1},
        \coord{Marcel}{Dupond}{1},
    \coord{Michelle}{Durand}{2},
    \coord{Marcelle}{Durand}{1}}

\adresse{\affil{1}{Laboratoire Traitement des Signaux \\
         1 rue de la Parole, BP 00000, 99000 Nouvelleville Cedex 00, France}
         \affil{2}{Laboratoire Traitement des Images \\
         1 rue de la Vision, BP 99999, 00000 Autreville, France}}

%% Si tous les auteurs ont la m�me adresse %%%%%%%%%%%%%%%%%%%%%%%%%%%%%%%%%%%%
%                                                                             %
%   \auteur{\coord{Michel}{Dupont}{},                                         %
%           \coord{Marcel}{Dupond}{},                                         %
%           \coord{Michelle}{Durand}{},                                       %
%           \coord{Marcelle}{Durand}{}}                                       %
%                                                                             %
%   \adresse{\affil{}{Laboratoire Traitement des Signaux et des Images \\     %
%     1 rue de la Science, BP 00000, 99999 Nouvelleville Cedex 00, France}}   %
%                                                                             %
%%%%%%%%%%%%%%%%%%%%%%%%%%%%%%%%%%%%%%%%%%%%%%%%%%%%%%%%%%%%%%%%%%%%%%%%%%%%%%%

\email{Michel.Dupont@labo.nouvelleville.fr,
Marcel.Dupond@labo.nouvelleville.fr\\
Michelle.Durand@ailleurs.fr, Marcelle.Durand@ailleurs.fr}

\resumefrancais{Les auteurs publiant au GRETSI et utilisant le
traitement de texte \LaTeXe\ trouveront ci-dessous quelques
indications destin{\'e}es {\`a} leur faciliter la t{\^a}che. Le
fichier \texttt{gretsifr.tex} qui contient le pr{\'e}sent
document respecte les contraintes fix{\'e}es ; recopiez le, par
exemple sous le nom \texttt{monarticle.tex}, et placez votre
texte aux endroits appropri{\'e}s.}

\resumeanglais{GRETSI authors who are \LaTeXe\ users will find
above some informations to help them. The file
\texttt{gretsifr.tex} which contains this document obeys the
rules; copy it, with the name \texttt{mypaper.tex} for instance,
and put your text in appropriate fields.}

\begin{document}
\maketitle

\section{Format du document}
\subsection{La classe \texttt{gretsi}}
Votre article ne doit pas d\'epasser 4 pages, tableaux et figures
inclus. Il est constitu{\'e} de deux colonnes de 88~mm,
espac{\'e}es de 6~mm. La classe \texttt{gretsi.cls} au format
\LaTeXe\ que nous vous recommandons d'utiliser vous permettra de
r{\'e}aliser automatiquement la mise en page, {\`a} l'aide de la
commande :
\begin{verbatim}
    \documentclass{gretsi}
\end{verbatim}
Dans le pr{\'e}ambule de votre fichier, vous devrez alors entrer les
informations suivantes :
\begin{itemize}

    \item le titre de l'article :\\
    \verb!\titre{Titre de l'article}!

    \item le pr{\'e}nom et le nom de chaque auteur, suivi d'un num{\'e}ro
    renvoyant {\`a} son adresse :\\
    \verb!\auteur{\coord{Pierre}{Dupont}{1},!\\
    \verb!        \coord{John}{Smith}{2}}!

    \item l'adresse de chaque auteur :\\
    \verb!\adresse{\affil{1}{Laboratoire \\! \\
    \verb!         rue, ville, France}     ! \\
    \verb!         \affil{2}{Laboratoire \\! \\
    \verb!         rue, ville, France}     !

    \item l'adresse {\'e}lectronique des auteurs :\\
    \verb!\email{Prenom.Nom@labo.fr, pnom@ecole.fr}!

    \item les r{\'e}sum{\'e}s en fran\c{c}ais et en anglais :\\
    \verb!\resumefrancais{R\'esum\'e fran\c{c}ais}! \\
    \verb!\resumeanglais{English written abstract}!

    \item enfin, le texte de votre article, et votre bibliographie : \\
    \verb!\begin{document}! \\
    \verb!\maketitle! \\
    \verb!Texte de l'article! \\
    \verb!\begin{thebibliography}{99}! \\
    \verb!Les r{\'e}f{\'e}rences! \\
    \verb!\end{thebibliography}! \\
    \verb!\end{document}!

\end{itemize}

\subsection{Titre et sous-titres}
Ce document utilise la commande \verb!\section! et la commande
\verb!\subsection!. Plus bas dans la hi{\'e}rarchie, voici ce qui est
obtenu :
\subsubsection{Sous-sous-titre}
\`A l'aide de la commande \verb!\subsubsection!.
\paragraph{Sous-sous-sous-titre}
\`A l'aide de \verb!\paragraph!.

\section{Langue et typographie}
Votre article {\'e}tant sans doute r{\'e}dig{\'e} en fran\c{c}ais, il
est pr{\'e}f{\'e}rable d'utiliser les r{\`e}gles de typographie
fran\c{c}aises. Si votre moteur \LaTeX\ est francis{\'e}, nous vous
conseillons d'utiliser le paquet \texttt{babel}, ou le paquet
\texttt{french}. 
\newpage
\subsection{Le paquet \texttt{french}}
Il suffit d'ins{\'e}rer dans le pr{\'e}ambule de votre fichier la commande :
\begin{verbatim}
    \usepackage{french}
\end{verbatim}
Si vous utilisez d'autres paquets (par exemple pour l'insertion de
graphiques), l'auteur du paquet \texttt{french.sty} conseille de faire
appara{\^\i}tre la commande ci-dessus en dernier dans la liste des
commandes \verb!\usepackage!.

\subsection{Le paquet \texttt{babel}}
Moins d{\'e}velopp{\'e} que le pr{\'e}c{\'e}dent, il s'utilise en
ins{\'e}rant dans le pr{\'e}ambule de votre fichier la commande :
\begin{verbatim}
    \usepackage[english,francais]{babel}
\end{verbatim}
Ceci indique que le language par d{\'e}faut est le fran\c{c}ais.
L'anglais est n{\'e}cessaire pour le texte de votre r{\'e}sum{\'e} en
anglais. Il est {\'e}galement possible d'entrer :
\begin{verbatim}
    \usepackage[english,french]{babel}
\end{verbatim}
Dans ce cas, si vous disposez aussi du paquet \texttt{french}, il sera
charg{\'e} automatiquement.

\subsection{Saisie}
La ponctuation haute (\verb&; : ? !&) doit {\^e}tre
pr{\'e}c{\'e}d{\'e}e d'un espace, comme sur une machine {\`a}
{\'e}crire. Les guillemets {\`a} la fran\c{c}aise sont entr{\'e}s de
la fa\c{c}on suivante :
\begin{itemize}
    \item \verb!\og{}texte entre guillemets\fg{}! avec \texttt{babel};
    \item \verb!<< texte entre guillemets >>!, ou encore \\
    \verb!\leftguillemets{}texte\rightguillemets{}! \\
    avec \texttt{french} 
\end{itemize}
Enfin, les majuscules doivent {\^e}tre accentu{\'e}es.

\section{Tableaux, figures \\ et math{\'e}matiques}
Les tableaux doivent {\^e}tre pr{\'e}c{\'e}d{\'e}s par leur
l{\'e}gende, comme cela est fait pour le tableau
\ref{puissancededeux}.
\begin{table}[htb]
    \legende{\label{puissancededeux}puissance de 2}
    \begin{center}
    \begin{tabular}{||c||*{8}{c|}|}
        \hline\hline
        $n$   & 1 & 2 & 3 &  4 &  5 &  6 &   7 &   8 \\ \hline
        $2^n$ & 2 & 4 & 8 & 16 & 32 & 64 & 128 & 256 \\
        \hline\hline
    \end{tabular}
    \end{center}
\end{table}
La l{\'e}gende est entr{\'e}e {\`a} l'aide de la commande
\verb!\legende!, qui remplace la commande \verb!\caption! habituelle
de \LaTeXe\ (de fa\c{c}on {\`a} harmoniser le r{\'e}sultat produit par
les paquets \texttt{french} et \texttt{babel}).

\`A l'inverse, les figures doivent {\^e}tre suivies par leur titre,
comme c'est le cas de la figure \ref{cercle}.
\begin{figure}[htb]
    \begin{center}
    \setlength{\unitlength}{0.5cm}
    \begin{picture}(5,5)
        \put(2.5,2.5){\oval(5,5)}
        \put(1,1){\line(1,0){3}}
        \put(4,1){\line(0,1){3}}
        \put(1,4){\line(1,0){3}}
        \put(1,1){\line(0,1){3}}
    \end{picture}
    \end{center}
    \legende{\label{cercle}un carr{\'e} dans un ovale}
\end{figure}

L'insertion de figure \texttt{PostScript} peut {\^e}tre faite
efficacement {\`a} l'aide des paquets \texttt{graphics},
\texttt{graphicx} ou \texttt{epsfig}. Pour ins{\'e}rer le fichier
\texttt{fig.eps}, si l'on veut que la largeur de la figure s'adapte
{\`a} la largeur de la colonne, il faut entrer, pour le paquet
\texttt{graphics}, les commandes suivantes :
\begin{verbatim}
    \begin{figure}[htb]
    \begin{center}
    \resizebox{88mm}{!}{
    \includegraphics{fig.eps}}
    \end{center}
    \legende{titre de la figure}
    \end{figure}
\end{verbatim}
Avec \texttt{graphicx}, il faut entrer :
\begin{verbatim}
    \begin{figure}[htb]
    \begin{center}
    \includegraphics[width=88mm]{fig.eps}
    \end{center}
    \legende{titre de la figure}
    \end{figure}
\end{verbatim}
Avec \texttt{epsfig}, il faut entrer :
\begin{verbatim}
    \begin{figure}[htb]
    \begin{center}
    \epsfig{file=fig.eps,width=88mm}
    \end{center}
    \legende{titre de la figure}
    \end{figure}
\end{verbatim}
Quant aux formules math{\'e}matiques, leur aspect pourra {\^e}tre
am{\'e}lior{\'e} en utilisant le paquet \texttt{amsmath} de la
librairie $\mathcal{AMS}$-\LaTeX. Elles seront num\'erot\'ees, comme c'est
le cas de la formule \ref{formule} :
\begin{equation}
   \label{formule}
   F(x) = \int_{-\infty}^x f(u)\,du
\end{equation}

\begin{thebibliography}{99}

\bibitem{companion}
M.~Goossens, F~Mittelbach et A.~Samarin.
\emph{The \LaTeX{} Companion}.
Addison-Wesley, 1994.

\bibitem{lamport94a}
L.~Lamport.
\emph{\LaTeX{} User's Guide and Reference Manual}.
Addison-Wesley, 1994.

\end{thebibliography}

\end{document}
