\documentclass{gretsi}
  
  
\usepackage[english,french]{babel}   % "babel.sty" + "french.sty"

% \usepackage{french}                  % "french.sty"
\usepackage{times}			% ajout times le 30 mai 2003
\usepackage{subfig}
\usepackage{graphicx}
\usepackage[justification=centering]{caption}
\usepackage{amsmath}
\usepackage[linesnumbered,ruled]{algorithm2e}
%\usepackage{enumitem}
%% --------------------------------------------------------------
%% FONTS CODING ?
% \usepackage[OT1]{fontenc} % Old fonts
\usepackage[T1]{fontenc}  % New fonts (preferred)
%% ==============================================================

\title{SIFT descriptor to set landmarks on biological images}

\author{\coord{Van Linh}{LE}{1,3},
        \coord{Marie}{BEURTON-AIMAR}{1},
    \coord{Adrien}{KRAHENBUHL}{1},
    \coord{Nicolas}{PARISEY}{2}}

\address{\affil{1}{Laboratoire Bordelais de Recherche en Informatique (LaBRI) \\
         351, cours de la Lib�ration, F-33405 Talence cedex, France}
         \affil{2}{Institute for genetics, the environment and plant protection (IGEPP), INRA 1349  \\
         35653, Le Rheu, France}
         \affil{3}{Faculty of Information Technology, Dalat university  \\
         1 Phu Dong Thien Vuong, Dalat, Lamdong, Vietnam}}

%% If all authors have the same address %%%%%%%%%%%%%%%%%%%%%%%%%%%%%%%%%%%%%%%
%                                                                             %
%   \auteur{\coord{Michel}{Dupont}{},                                         %
%           \coord{Marcel}{Dupond}{},                                         %
%           \coord{Michelle}{Durand}{},                                       %
%           \coord{Marcelle}{Durand}{}}                                       %
%                                                                             %
%   \adress{\affil{}{Laboratoire Traitement des Signaux et des Images \\      %
%     1 rue de la Science, BP 00000, 99999 Nouvelleville Cedex 00, France}}   %
%                                                                             %
%                                                                             %
%%%%%%%%%%%%%%%%%%%%%%%%%%%%%%%%%%%%%%%%%%%%%%%%%%%%%%%%%%%%%%%%%%%%%%%%%%%%%%%

\email{van-linh.le,beurton,adrien.krahenbuhl@labri.fr, nparisey@rennes.inra.fr}


\frenchabstract{L'analyse d'images est une des �tapes majeures du
  traitement d'images num�riques, on l'applique aussi bien en imagerie
m�dicale ou biologique, en vision par ordinateur ... En biologie, les
  images sont tr�s utilis�es, entre autre pour analyser la structure
  ou le comportement de mol�cules, les caract�ristiques des tissus ou
  encore pour mesurer et classifier des d�tails anatomiques. La
  classification d'�chantillons biologiques peut �tre obtenue � partir
  d'�tudes morphologiques mais qui sont jusqu'� pr�sent tr�s largement
  r�alis�es manuellement. Pour ce travail nous avons pr�t� attention
  au probl�me de la d�finition de landmarks, points d'int�r�t dans une
  image, et � l'estimation automatique de leur position. Pour cela,
  nous avons travaill� sur une utilisation particuli�re du descripteur
  SIFT pour identifier pr�cis�ment ces landmarks dans une image cible,
  � partir d'un jeu de landmarks manuels position�s manuellement dans
  une image source. Nous avons pour cela r�duit d'une part la matrice
  de calcul du descripteur et d'autre part d�fini une zone de
  recherche dans l'image cible au lieu d'appliquer la recherche �
  toute l'image. Notre chaine de traitemnet qui contient par ailleurs
  une segmentation et une �tape d'alignement des images, a �t� test�e
  sur un ensemble de 290 images de mandibules de carabes pour lesquels
  nous avons dispos� d'une v�rit� terrain consistant en un jeu de
  landmarks manuels fourni par les biologistes de l'INRA pour chaque image.
  La plateforme de traitement appel�e MAELab est disponible � l'heure actuelle sous
  forme de librairie librement accessible sur github.}

\englishabstract{Image analysis is a large field in image
  processing and it has applied in practice with many application in
  the different majors such as medicine, machine vision, biology,
  ... In biology, images are widely used from a long time, to study
molecule structures or behaviors, tissus characteristics and in
general to measure and to classify anatomical details. Classification
of biological samples can be obtained from studying morphological
features but at this time, setting morphological markers is done
manually. In this work, we have focused on the problem to replace
manual operations by automatic procedures to set landmarks which are
point of interests in biological images. This paper presents how we
  have designed a specific way to use the SIFT descriptor to improve
  the results that we have obtained before to achieve this task. The
  two main characteristics of our method are the reduction of the
  patch area to compute de source descriptor and the definition of a
  restrictive search area of the target. The efficiency of
the method is evaluated on two set of images: left and right mandibles
of beetles belonging to a study of the national institute of
  agriculture (INRA). The complete method is implemented in a
  framework called  MAELab and freely
available as a C++ library.}

\begin{document}
\maketitle

\section{Introduction}
Morphometry analysis is an important field of image analysis in
biology. It is used to characterize the shape variations of the
organisms. From obtained information, the biologists can evaluate the
evolution of an organism or detect differences between several
ones. Depending on the requirements of the application, the
output of analysis process can be measures of shape, color ... or
identificaton of pattern or landmark (points of interest) positions. Landmarks are
points along an image outline that store a lot of important
information about the shape of the image. The morphometric landmarks
are precise points defined by the biologists. They are used in many
biological studies and included into the classification tasks. Until
now, the morphometric landmarks are mainly manually identified. The
manual identification is time-consuming and could vary a lot depending on
the operator.

\begin{figure}[h]
\centering
\subfloat[Left mandible]{\label{figrbox2}\includegraphics[width=0.17\textwidth]{./images/lm}}~~
\subfloat[Right mandible]{\label{figrbox1}\includegraphics[width=0.17\textwidth]{./images/rm}}
\caption{Example of beetle mandibles from the studied data set with
  manual landmarks.}
\label{fig1}
\end{figure}

In this paper, we focus on a specific part of our framework which
consists to introduce the computing of SIFT descriptors. This information
about landmark areas allows to reduce the size of the source patch and the
target search space at the same time.
These works are evaluated on a dataset corresponding to a collection of
293 beetles from Brittany lands. The first step of
these works concerns the left and right mandibles. For each beetle, a
set of landmarks has been manually positioned by the biologists (see Fig. 1). In our
study, we used this dataset as ground truth to evaluate the position
of the automatically estimated landmarks. Our framework also contains
steps of segmentation, registration, and 
descriptors comparison. Each part will be discussed on the
experiments results.
\section{Landmark descriptor}
For each beetle, the morphometric landmarks have been manually
set on mandibles images by the biologists: 16 and 18 landmarks
for each left and right mandible, respectively. The considered
problem is the automatic detection of these landmarks on a mandible
image to replace the manual one. In the whole of the process, 
the landmarks will be estimated on a target image by using the 
manual landmarks of a source image. The source image is 
chosen randomly from the set of all images.

In this section, firstly, we have a summary about the general
usage of SIFT. Then, we will discuss about the details of our method of
landmark estimation.

\subsection{The adapted SIFT method}
\label{secSIFT}
The SIFT method has been proposed by D. Lowe \cite{lowe1999object,
  lowe2004distinctive}. It is used to extract distinctive features
from the images, potentially invariant to rotation, scale and noise.
The SIFT features can be used to
determine a matching between source and target images.
The SIFT method is composed by four steps: (1)
scale-space extrema detection, (2) keypoint localization, (3)
orientation assignment, and (4) keypoint descriptor.


In the first step, a difference of Gaussian (DoG)
\cite{davidson2006molecular}  function is applied to identify the
interest points at all scales and orientation of the input image. The keypoints
are taken as the maximal and minimal of the result of DoG function computed at
multiple scales. The scale-space extrema detection produces a lot of keypoint
candidates, with some of them are unstable. In the second step of SIFT,
the key point candidates are localized and refined by suppressing the ones which 
have the low contrast or are poorly localized along an edge.

Then, the orientation and gradient magnitude of key points are
calculated by considered their 4-neighborhoods. Finally, the descriptor is computed 
for each key point, from the orientation and gradient magnitude. 
A descriptor is a region of
$16 \times 16$ pixels around the key point.

By applying the original SIFT into our problem, we have succeeded
indicating the keypoints in the image (see Fig. \ref{fig3}). But we do
not have the result when we try to extract keypoint correspondences 
between the source and target image. The problem is carried from
choosing the best points from the large set of the candidates.
To solve this problem, we have modified the method in order to limit
the search space before applying the SIFT computing. We have also
change the size of the gird to calculate the
SIFT descriptor.
\begin{figure}[htb]
    \centering
    \includegraphics[width=0.25\textwidth]{./images/siftc}
    \caption{SIFT keypoints in a right mandible.}
    \label{fig3}
\end{figure}
\subsection{SIFT using into landmark area}
As mentioned in section \ref{secSIFT}, we have tried to use the standard
version of SIFT to detect the landmarks. But the result contains a lot of
candidates for the estimated landmarks. Moreover, when we compare the position of
the detected points with the ground truth (manual landmarks), they are
very far. Last, searching in whole image is very time
consuming. To obtain better result, we propose to exploit the SIFT
descriptors by another way.


Firstly, two images are segmented and two lists of
contours points are registered. Then, the area (called patch) around each
source landmark is defined and a larger patch is extracted in the
target image at the same position. The SIFT descriptor is then calculated
as usual. But we have reduced two first steps of SIFT and
have changed the size of the sample regions for $9 \times 9$ (instead of
$16 \times 16$). This size has been empirically determined after several tests. The
SIFT descriptor for the patch is also an histogram containing the sum
of pixel gradients for each consider direction. The comparison between
two SIFT-descriptors is done by an L2-distance as in Eq. \ref{eql2}:

\begin{equation}
\label{eql2}
	L(D1,D2) = \sum\limits_{i = 0}^{n}\sqrt{(D1_i-D2_i)^2}
\end{equation}
Where:
\begin{itemize}%[nosep,label=\footnotesize$\bullet$]
	\item $n$ is the number of directions
	\item $D1$ and $D2$ are two descriptors of size $n$,
	\item $D1_i$ and $D2_i $ are the $i^{th}$ descriptor values.
\end{itemize}
Our approach to use the SIFT descriptors into our work is presented in
Fig. \ref{fig4}. To detect the target landmarks, a registration is
computed between the source and target images. This step is described in
the experiment results. Then, the patch $P_s$
of the source and $P_t$ of the target are created with the size of $P_s$
smaller than the size of $P_t$. For each pixel in $P_t$, a
sub-patch $P_t'$ is extracted with the same size of $P_s$. When the
$P_t'$ have a part outside of $P_t$, the pixels outside the
patch will be considered. Then, the distance $L(P_s,P_t')$ is computed
following Eq. (\ref{eql2}). The process ends when all the pixels in
patch $P_t$ are considered. The position of the estimated landmark
corresponds to the position of the sub-patch $P_t'$ giving the smallest
distance $L(P_s,P_t')$. 
%Finally, the position of the estimated landmark is set in the original position of the target image.

\begin{figure}[htb]
    \centering
    \includegraphics[width=0.35\textwidth]{./images/illustration_SIFT}
    \caption{Steps of SIFT descriptors comparisons between the patch $P_s$ of the source image and the patches $P'_t$ of the target image.}
    \label{fig4}
\end{figure}

By experiments, a patch sample of $9 \times 9$ pixels centered in each
landmark on the source and the size of $36 \times 36$ is kept for the
patch on the target image.

\section{Experiments and results}
The method is tested on two sets (left and right) of
beetle mandibles. After verifying the dataset and suppressing of same unusable images (broken or containing a hidden mandible), it remains 290 images of right
mandibles and 286 left mandible images. In all valid images, the position
of a set of manual landmarks is indicated by biologists: 18 for each right
mandible, 16 for each left mandible.


As discussed in previous section, before applying the SIFT descriptor to estimate the
landmarks on the target image, we have to estimate the search space
containing the landmarks.


Firstly, the segmentation is applied on both on source and target
image. The Canny algorithm \cite{canny1986computational} was chosen to
perform this step. To use the Canny algorithm, two thresholds value
must be provided ($T_{lower}, T_{upper}$). As mentioned in
\cite{adaptiveCanny}, defining the threshold values is a difficult
problem. If the threshold values are unsuitable, the contours could be
far with ground truth (fewer contours or more noise). In our case, the
lower threshold value is determined by analysis of the image histogram \cite{leestimating}. The ratio of two threshold values that we have chosen is $1:3$ to consider a wide range of the values. During the Canny
computing, the direction of the gradient of each pixel belonging to
the contours is kept in order to be used later. At the end of the 
segmentation step, a simple algorithm is applied to remove edges
inside the main contours.


Then, a Principal Component Analysis (PCA) is applied to register 
two lists of contours points from the source and the target \cite{shlens2014tutorial, jolliffe2002principal}. 
The lists of contour points are used  as input. 
For each list of contours points, the centroid point 
and principal axis are computed. The centroid point coordinates are computed
like the average coordinate of all contours points. The principal axis is
a line connecting the centroid point to a contour point which has the minimum of average perpendicular distance to remaining contour points. Then, two lists of contours points are registered by
computing the translation and rotation parameter values. 
The translation is computed as the distance between the centroid points of the source and the target. The rotation angle is the angle between the principal axes of two images. However in some case, 
the result of the segmentation step could contains noise,
affecting to the registration step. To improve the registration,
we have built up an iterative PCA until stabilization
(PCAI). Finally, the SIFT descriptor allows to determine the
estimated landmark on the target image.


We have run the method on all usable images. As results, the estimated
landmarks are well positioned on the major part of targets but not in all.
As precedingly raised, the mandible images can have different sizes
due to the different sizes of beetles. We detected that our method is sensible to these differences.
To improve the results, we have inserted a pre-process before the
computing of the SIFT descriptor to estimate the difference of scale between the source
and target images. The bounding box of the mandible contours in the source and target images
is defined by checking the coordinate of the contour points. The scale
between two images is defined as the ratio of the two bounding box sizes.


The Fig. \ref{fig5} shows the final result for a right and a left
mandible with manual and estimated landmarks. The estimated landmarks
are quite near with the manual ones, as conformed by the following statistics.


\begin{figure}[h]

\centering
\subfloat[Left mandible]{\label{figrbox2}\includegraphics[width=0.17\textwidth]{./images/mg_rs}}~~
\subfloat[Right mandible]{\label{figrbox1}\includegraphics[width=0.17\textwidth]{./images/md_rs}}
\caption{The manual (in red) and estimated (in yellow) landmarks on a mandible.}
\label{fig5}
\end{figure}

First statistics are done on the mean accuracy of all landmarks on
the target images. The error is computed as the distance between the
manual and the corresponding estimated landmark on the target image with
an accepted error from 1\% to 2\% of the bounding box's size (when we
consider the scale of the image). According to this way, the results are
shown in Fig. \ref{fig6}. The score of well-positioned landmarks is 
\textbf{87.03\%} for the set of
right mandibles and \textbf{78.82\%} for left mandibles.

\begin{figure}[h]
\centering
\subfloat[Set of right mandibles.]{\label{figmdresult}\includegraphics[width=0.18\textwidth]{./images/mdresult}}~~
\subfloat[Set of left mandibles.]{\label{figmgresult}\includegraphics[width=0.18\textwidth]{./images/mgresult}}
\caption{The mean proportion of well and bad landmark locations of the two sets of left and right mandibles.}
\label{fig6}
\end{figure}

Beside the global results, we are also interested by the individual accuracy of
each estimated landmark. The error measurement is the same with the first
one (distance between manual landmark and the corresponding estimated landmark) but
the acceptance is done with the standard deviation of the distances
\cite{bland1996statistics}. The Fig. \ref{fig7} and \ref{fig8} show the
proportion of well estimated landmarks on each dataset. With 18
landmarks of right mandible, the highest proportion is obtained by the
$1^{st}$ landmark with \textbf{98.62\%}; the lowest proportion is
\textbf{74.48\%} for the $14^{th}$ landmark. The remaining landmarks
are also estimated with a high accuracy greater than
\textbf{75\%}. For left mandibles, the highest and lowest success rates
are \textbf{93.01\%} for the $1^{st}$ landmark and \textbf{60.14\%}
for the $16^{th}$ landmark. In this evaluation, we can see that the
correct proportion on the $11^{th}$ and $12^{th}$ landmark of the left
mandible and the $13^{th}$ and $14^{th}$ landmark of the right
mandible are less than other landmarks. This is due to the noise of 
the contours at the base of mandibles higher than on the top part. 
The software MAELab proposes an implementation of the SIFT version
for the automatic landmarks estimation. 
It is written in C++ and distributed as
free library on the Github.

\begin{figure}[htb]
    \centering
    \includegraphics[width=0.43\textwidth]{./images/md_chartlms}
    \caption{The proportions of well estimated landmarks for each model landmark of right mandibles.}
    \label{fig7}
\end{figure}
\begin{figure}[htb]
    \centering
    \includegraphics[width=0.43\textwidth]{./images/mg_chartlms}
    \caption{The proportions of well estimated landmarks for each model landmark of left mandibles.}
    \label{fig8}
\end{figure}
\section{Conclusion and discussion}
The landmark positioning is the main way for the image analysis analysis in biology.
In this paper, we presented a solution based on the SIFT
descriptors for the segmentation and registration processes of 
the landmark estimation on beetle mandibles. Firstly, each mandible has been
segmented and its contours is extracted. The SIFT descriptor
is computed on each contour point to find the best matching position of estimated
landmarks. The results show that this new method succeeds in locating all
landmarks in a request image. The accuracy of the method is sufficient
to be proposed to biologists as a replacement of the manual
positioning. Moreover, considering the previous work
\cite{leestimating}, the precision of the position of the estimated landmarks has been
improved. The next step consists to increase the number of well positioned landmarks in the bottom of the
mandible.


\bibliographystyle{plain}
\bibliography{references}

\end{document}
